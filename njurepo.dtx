% \iffalse meta-comment
%
% Copyright (C) 2019 by Zangwei Zheng <zhengzangw@gmail.com>
%
% This file may be distributed and/or modified under the conditions of
% the LaTeX Project Public License, either version 1.3c of this license
% or (at your option) any later version. The latest version of this 
% license is in:
%
%   http://www.latex-project.org/lppl.txt
%
% and version 1.3c or later is part of all distributions of LaTeX
% version 2005/12/01 or later.
%
% \fi
%
% \iffalse
%<*driver>
\ProvidesFile{njurepo.dtx}[2019/01/25 1.0.0 Nanjing University Report Template]
\documentclass{ltxdoc}
\usepackage{dtx-style}
    \EnableCrossrefs
    \CodelineIndex
    \RecordChanges
\begin{document}
    \DocInput{njurepo.dtx}
    \PrintChanges
    \PrintIndex
\end{document}
%</driver>
% \fi
%
% \CheckSum{0}
%
% \CharacterTable
%  {Upper-case    \A\B\C\D\E\F\G\H\I\J\K\L\M\N\O\P\Q\R\S\T\U\V\W\X\Y\Z
%   Lower-case    \a\b\c\d\e\f\g\h\i\j\k\l\m\n\o\p\q\r\s\t\u\v\w\x\y\z
%   Digits        \0\1\2\3\4\5\6\7\8\9
%   Exclamation   \!     Double quote  \"     Hash (number) \#
%   Dollar        \$     Percent       \%     Ampersand     \&
%   Acute accent  \'     Left paren    \(     Right paren   \)
%   Asterisk      \*     Plus          \+     Comma         \,
%   Minus         \-     Point         \.     Solidus       \/
%   Colon         \:     Semicolon     \;     Less than     \<
%   Equals        \=     Greater than  \>     Question mark \?
%   Commercial at \@     Left bracket  \[     Backslash     \\
%   Right bracket \]     Circumflex    \^     Underscore    \_
%   Grave accent  \`     Left brace    \{     Vertical bar  \|
%   Right brace   \}     Tilde         \~}
%
% \DoNotIndex{\newenvironment,\@bsphack,\@empty,\@esphack,\sfcode}
% \DoNotIndex{\addtocounter,\label,\let,\linewidth,\newcounter}
% \DoNotIndex{\noindent,\normalfont,\par,\parskip,\phantomsection}
% \DoNotIndex{\providecommand,\ProvidesPackage,\refstepcounter}
% \DoNotIndex{\RequirePackage,\setcounter,\setlength,\string,\strut}
% \DoNotIndex{\textbackslash,\texttt,\ttfamily,\usepackage}
% \DoNotIndex{\begin,\end,\begingroup,\endgroup,\par,\\}
% \DoNotIndex{\if,\ifx,\ifdim,\ifnum,\ifcase,\else,\or,\fi}
% \DoNotIndex{\let,\def,\xdef,\edef,\newcommand,\renewcommand}
% \DoNotIndex{\expandafter,\csname,\endcsname,\relax,\protect}
% \DoNotIndex{\Huge,\huge,\LARGE,\Large,\large,\normalsize}
% \DoNotIndex{\small,\footnotesize,\scriptsize,\tiny}
% \DoNotIndex{\normalfont,\bfseries,\slshape,\sffamily,\interlinepenalty}
% \DoNotIndex{\textbf,\textit,\textsf,\textsc}
% \DoNotIndex{\hfil,\par,\hskip,\vskip,\vspace,\quad}
% \DoNotIndex{\centering,\raggedright,\ref}
% \DoNotIndex{\c@secnumdepth,\@startsection,\@setfontsize}
% \DoNotIndex{\ ,\@plus,\@minus,\p@,\z@,\@m,\@M,\@ne,\m@ne}
% \DoNotIndex{\@@par,\DeclareOperation,\RequirePackage,\LoadClass}
% \DoNotIndex{\AtBeginDocument,\AtEndDocument}
%
% \changes{v1.0}{2019/01/22}{Initial version}
%
% \GetFileInfo{\jobname.dtx}
%
% \def\indexname{索引}
% \def\glossaryname{修改记录}
% \IndexPrologue{\section{\indexname}}
% \GlossaryPrologue{\section{\glossaryname}}

% \title{\bfseries\color{violet}\njurepo: 为了方便整合起来的模板}
% \author{郑奘巍 \\[5pt]\texttt{zhengzangw@gmail.com}}
% \date{\fileversion\ (\filedate)}
% \maketitle\thispagestyle{empty}
%
% \begin{abstract}\noindent
% 此宏包旨在建立一个免于配置的作业、实验报告模板。借鉴的宏包如下:
% \begin{itemize}
%  \item 清华大学\thuthesis
% \end{itemize}
% \end{abstract}
%
%
% \vskip2cm
% \def\abstractname{免责声明}
% \begin{abstract}
% \noindent
% \begin{enumerate}
% \item 本模板的发布遵守 \LaTeX\ Project Public License,使用前请认真阅读协议内
%   容。
% \item 本模板为作者根据清华大学的论文模板\thuthesis等改写而成。因作者能力有限,可以选择使用\thuthesis模板
% 并自定义选项和手动添加宏包,可能达到比本模板更加好的效果。
% \item 任何个人或组织以本模板为基础进行修改、扩展而生成的新的专用模板,请严格遵
%   守 \LaTeX\ Project Public License 协议。由于违犯协议而引起的任何纠纷争端均与
%   本模板作者无关。
% \end{enumerate}
% \end{abstract}
%
% \clearpage
% \pagestyle{fancy}
% \begin{multicols}{2}[
%   \setlength{\columnseprule}{.4pt}
%   \setlength{\columnsep}{18pt}]
%   \tableofcontents
% \end{multicols}
% \clearpage
%
%
% \section{安装}
% \label{sec:installation}
% \njurepo 尚未包含于\TeX 发行版中,需要自行前往github主页安装。相关链接:
% \begin{itemize}
% \item 主页: 有待添加
% \end{itemize}
%
% \subsection{模板的组成}
% 下表列出了\njurepo 的主要文件及其功能介绍:
%
% \begin{longtable}{l|p{8cm}}
% \toprule
% {\heiti 文件(夹)} & {\heiti 功能描述}\\\midrule
% \endfirsthead
% \midrule
% {\heiti 文件(夹)} & {\heiti 功能描述}\\\midrule
% \endhead
% \endfoot
% \endlastfoot
% njurepo.ins & \textsc{DocStrip} 驱动文件(开发用) \\
% njurepo.dtx & \textsc{DocStrip} 源文件(开发用)\\\midrule
% njurepo.cls & 模板类文件\\
% njurepo.cfg & 模板配置文件\\\midrule
% thuthesis.sty & 为示例文档加载其它宏包\\\midrule
% \textbf{njurepo.pdf} & 用户手册(本文档)\\\bottomrule
% \end{longtable}
%
% \section{使用说明}
% \subsection{选项}
% \label{sec:option}
% \DescribeOption{language}
% 论文的主要语言(默认:中文)。可选:\option{chinese},\option{english}。
% 
% \DescribeOption{secret}
% \DescribeOption{open}
% 尚未添加的选择
% \begin{itemize}
% \item secret: 同Thuthesis的设置
% \item open: any为页面连续,right为章节奇数页
% \end{itemize}
%
% \subsection{字体配置}
% \label{sec:font-config}
% 使用\CTeX\ 默认字体配置
% \subsubsection{字体命令}
% \label{sec:fontcmds}
% \myentry{字体}
% \DescribeMacro{\songti}
% \DescribeMacro{\fangsong}
% \DescribeMacro{\heiti}
% \DescribeMacro{\kaishu}
% 用来切换宋体、仿宋、黑体、楷体四种基本字体。
% \myentry{字号}
% \DescribeMacro{\chuhao}
% \DescribeMacro{\xiaochu}
% \DescribeMacro{\yihao}
% \DescribeMacro{\xiaoyi}
% \DescribeMacro{\bahao}
% 定义字体大小,分别为
% \begin{center}
% \begin{tabular}{llllll}
% \toprule
% \cs{chuhao} & \cs{xiaochu} & \cs{yihao}  & \cs{xiaoyi}    & \cs{erhao}  & \cs{xiaoer}\\
% \cs{sanhao} & \cs{xiaosan} & \cs{sihao}  & \cs{banxiaosi} & \cs{xiaosi} & \cs{dawu}\\
% \cs{wuhao}  & \cs{xiaowu}  & \cs{liuhao} & \cs{xiaoliu}   & \cs{qihao}  & \cs{bahao}\\\bottomrule
% \end{tabular}
% \end{center}
% 使用方法为:\cs{command}\oarg{num},其中 command 为字号命令,num 为行距。比
% 如 \cs{xiaosi}[1.5] 表示选择小四字体,行距 1.5 倍。写作指南要求表格中的字体
% 是 \cs{dawu},模板已经设置好了。
%
% \StopEventually{}
%
% \section{实现细节}
% \subsection{基本信息}
%    \begin{macrocode}
%<*cls>
\NeedsTeXFormat{LaTeX2e}
\ProvidesClass{njurepo}
%</cls>
%<cfg>\ProvidesFile{njurepo.cfg}
%<cls|cfg>[2019/01/25 1.0.0 Nanjing University Report Template]
%    \end{macrocode}
%
% \subsection{定义选项}
% \label{sec:defoption}
% 使用kvoptions宏包进行选项设置
%    \begin{macrocode}
%<*cls>
\hyphenation{NJU-repo}
\def\thuthesis{\textsc{NJUrepo}}
\RequirePackage{kvoptions}
\SetupKeyvalOptions{
    family=hw,
    prefix=hw@,
    setkeys=\kvsetkeys
}
\DeclareStringOption[chinese]{language}[chinese]
\DeclareStringOption[any]{open}[any]
\DeclareBoolOption{secret}
\DeclareBoolOption{draft}
\DeclareDefaultOption{\PassOptionsToClass{\CurrentOption}{ctexbook}}

\ProcessKeyvalOptions*
%    \end{macrocode}
%
% 检测选项是否合法
%    \begin{macrocode}
\newcommand\hw@validate@key[1]{%
  \@ifundefined{hw@\csname hw@#1\endcsname true}{%
    \ClassError{njurepo}{Invalid value '\csname hw#1\endcsname'}{}
    }{%
      \csname hw@\csname hw@#1\endcsname true\endcsname
    }
}
\newif\ifhw@chinese
\newif\ifhw@english
\hw@validate@key{language}
\newif\ifhw@any
\newif\ifhw@right
\hw@validate@key{open}
%    \end{macrocode}
% 
% 使用ctexbook宏包
%    \begin{macrocode}
\LoadClass[a4paper,openany,UTF8,zihao=-4,scheme=plain]{ctexbook}
%    \end{macrocode}
%
% \subsection{加载宏包}
% \label{sec:loadpkg}
% 用于开发的宏包
%    \begin{macrocode}
\RequirePackage{etoolbox}
\RequirePackage{ifxetex}
\RequirePackage{xparse}
\RequirePackage{amsmath}
\RequirePackage{graphicx}
\RequirePackage[labelformat=simple]{subcaption}
\RequirePackage{pdfpages}
\includepdfset{fitpaper=true}
\RequirePackage{CJKfntef}
\RequirePackage[amsmath, thmmarks, hyperref]{ntheorem}
\RequirePackage{array}
\RequirePackage{longtable}
\RequirePackage[sort&compress]{natbib}
\RequirePackage{booktabs}
%    \end{macrocode}
%
% 超链接插入
%    \begin{macrocode}
\RequirePackage{hyperref}
\ifxetex
  \hypersetup{%
    CJKbookmarks=true}
\else
  \hypersetup{%
    unicode=true,
    CJKbookmarks=false}
\fi
\hypersetup{%
  linktoc=all,
  bookmarksnumbered=true,
  bookmarksopen=true,
  bookmarksopenlevel=1,
  breaklinks=true,
  colorlinks=false,
  plainpages=false,
  pdfborder=0 0 0}	
\urlstyle{same}
\def\UrlBreaks{%
  \do\/%
  \do\a\do\b\do\c\do\d\do\e\do\f\do\g\do\h\do\i\do\j\do\k\do\l%
     \do\m\do\n\do\o\do\p\do\q\do\r\do\s\do\t\do\u\do\v\do\w\do\x\do\y\do\z%
  \do\A\do\B\do\C\do\D\do\E\do\F\do\G\do\H\do\I\do\J\do\K\do\L%
     \do\M\do\N\do\O\do\P\do\Q\do\R\do\S\do\T\do\U\do\V\do\W\do\X\do\Y\do\Z%
  \do0\do1\do2\do3\do4\do5\do6\do7\do8\do9\do=\do/\do.\do:%
  \do\*\do\-\do\~\do\'\do\"\do\-}
\Urlmuskip=0mu plus 0.1mu
%    \end{macrocode}
%
% 页眉页脚设置
%    \begin{macrocode}
\RequirePackage{fancyhdr}
\RequirePackage{notoccite}	
%    \end{macrocode}
%
% \subsection{页面设置}
% 使用了hwthesis的非本科生默认配置。
%    \begin{macrocode}
\RequirePackage{geometry}
\geometry{
    a4paper, %210*297mm
    hcentering,
    ignoreall,
    nomarginpar,
    left=30mm,
    headheight=5mm,
    headsep=5mm,
    textheight=237mm,
    bottom=29mm,
    footskip=6mm
}
%    \end{macrocode}
%
% \subsection{主文档格式}
% \label{sec:mainbody}
%
% \begin{macro}{\cleardoublepage}
%    \begin{macrocode}
\let\hw@cleardoublepage\cleardoublepage
\newcommand{\hw@clearemptydoublepage}{%
  \clearpage{\pagestyle{hw@empty}\hw@cleardoublepage}}
\let\cleardoublepage\hw@clearemptydoublepage
%    \end{macrocode}
% \end{macro}
%
% \begin{macro}{\frontmatter}
% \begin{macro}{\mainmatter}
% \begin{macro}{\backmatter}
%    \begin{macrocode}
\renewcommand\frontmatter{%
    \ifhw@right\cleardoublepage\else\clearpage\fi
    \@mainmatterfalse
    \pagenumbering{Roman}
    \pagestyle{hw@empty}}
\renewcommand\mainmatter{%
    \ifhw@right\cleardoublepage\else\clearpage\fi
    \@mainmattertrue
    \pagenumbering{arabic}
    \pagestyle{hw@headings}}
\renewcommand\backmatter{%
    \ifhw@right\cleardoublepage\else\clearpage\fi
    \@mainmattertrue}
%    \end{macrocode}
% \end{macro}
% \end{macro}
% \end{macro}
%
% \subsection{字体与字号}
% \label{sec:font}
% \subsubsection{英文字体}
% 配置英文字体。
%    \begin{macrocode}
\newcommand\hw@fontset{\csname g__ctex_fontset_tl\endcsname}
\ifthenelse{\equal{\hw@fontset}{fandol}}{
  \setmainfont[
    Extension      = .otf,
    UprightFont    = *-regular,
    BoldFont       = *-bold,
    ItalicFont     = *-italic,
    BoldItalicFont = *-bolditalic,
  ]{texgyretermes}
  \setsansfont[
    Extension      = .otf,
    UprightFont    = *-regular,
    BoldFont       = *-bold,
    ItalicFont     = *-italic,
    BoldItalicFont = *-bolditalic,
  ]{texgyreheros}
  \setmonofont[
    Extension      = .otf,
    UprightFont    = *-regular,
    BoldFont       = *-bold,
    ItalicFont     = *-italic,
    BoldItalicFont = *-bolditalic,
    Scale          = MatchLowercase,
  ]{texgyrecursor}
}{
  \setmainfont{Times New Roman}
  \setsansfont{Arial}
  \ifthenelse{\equal{\hw@fontset}{mac}}{
    \setmonofont[Scale=MatchLowercase]{Menlo}
  }{
    \setmonofont[Scale=MatchLowercase]{Courier New}
  }
}
%    \end{macrocode}
%
% \subsubsection{数学环境字体}
% 配置数学字体(使用unicode-math)
%    \begin{macrocode}
\RequirePackage{unicode-math}
\unimathsetup{
  math-style = ISO,
  bold-style = ISO,
  nabla      = upright,
  partial    = upright,
}
\IfFontExistsTF{STIX2Math.otf}{
  \setmathfont[StylisticSet=8]{STIX2Math.otf}
  \setmathfont[range={scr,bfscr},StylisticSet=1]{STIX2Math.otf}
  \IfFontExistsTF{XITSMath-Regular.otf}{
    \setmathfont[range={\partial,\lbrace,\rbrace}]{XITSMath-Regular.otf}
  }{
    \setmathfont[range={\partial,\lbrace,\rbrace}]{xits-math.otf}
  }
}{
  \setmathfont[
    Extension    = .otf,
    BoldFont     = *bold,
    StylisticSet = 8,
  ]{xits-math}
  \setmathfont[range={cal,bfcal},StylisticSet=1]{xits-math.otf}
}
%    \end{macrocode}
%
% \subsubsection{数学环境符号}
% \begin{macro}{\ldots}
% 省略号一律居中,所以 \cs{ldots} 不再居于底部。
%    \begin{macrocode}
\ifhw@chinese
  \def\mathellipsis{\cdots}
\fi
%    \end{macrocode}
% \end{macro}
%
% \begin{macro}{\le}
% \begin{macro}{\ge}
% \begin{macro}{\leq}
% \begin{macro}{\geq}
% 小于等于号要使用倾斜的形式。
%    \begin{macrocode}
\protected\def\le{\leqslant}
\protected\def\ge{\geqslant}
\AtBeginDocument{%
  \renewcommand\leq{\leqslant}%
  \renewcommand\geq{\geqslant}%
}
%    \end{macrocode}
% \end{macro}
% \end{macro}
% \end{macro}
% \end{macro}
%
% \begin{macro}{\int}
% 积分号 \cs{int} 使用正体,并且上下限默认置于积分号上下两侧。
%    \begin{macrocode}
\removenolimits{%
  \int\iint\iiint\iiiint\oint\oiint\oiiint
  \intclockwise\varointclockwise\ointctrclockwise\sumint
  \intbar\intBar\fint\cirfnint\awint\rppolint
  \scpolint\npolint\pointint\sqint\intlarhk\intx
  \intcap\intcup\upint\lowint
}
%    \end{macrocode}
% \end{macro}
%
% \begin{macro}{\Re}
% \begin{macro}{\Im}
% \begin{macro}{\nabla}
% 实部、虚部操作符使用罗马体 $\mathrm{Re}$、$\mathrm{Im}$ 而不是 fraktur 体
% $\Re$、$\Im$。\cs{nabla} 使用粗正体。
%    \begin{macrocode}
\AtBeginDocument{%
  \renewcommand{\Re}{\operatorname{Re}}%
  \renewcommand{\Im}{\operatorname{Im}}%
  \renewcommand\nabla{\mbfnabla}%
}
%    \end{macrocode}
% \end{macro}
% \end{macro}
% \end{macro}
%
% \begin{macro}{\bm}
% \begin{macro}{\boldsymbol}
% 兼容旧的粗体命令:\pkg{bm} 的 \cs{bm} 和 \pkg{amsmath} 的 \cs{boldsymbol}。
%    \begin{macrocode}
\newcommand\bm{\symbf}
\renewcommand\boldsymbol{\symbf}
%    \end{macrocode}
% \end{macro}
% \end{macro}
%
% \begin{macro}{\square}
% 兼容 \pkg{amssymb} 中的命令。
%    \begin{macrocode}
\newcommand\square{\mdlgwhtsquare}
%    \end{macrocode}
% \end{macro}
%
% 允许太长的公式断行、分页等。
%    \begin{macrocode}
\allowdisplaybreaks[4]
\renewcommand\theequation{\ifnum \c@chapter>\z@ \thechapter-\fi\@arabic\c@equation}
%    \end{macrocode}
%
% 公式距前后文的距离由 4 个参数控制,参见 \cs{normalsize} 的定义。
%    \begin{macrocode}
\def\make@df@tag{\@ifstar\hw@make@df@tag@@\make@df@tag@@@}
\def\hw@make@df@tag@@#1{\gdef\df@tag{\hw@maketag{#1}\def\@currentlabel{#1}}}
\def\hw@maketag#1{\maketag@@@{(\ignorespaces #1\unskip\@@italiccorr)}}
\def\tagform@#1{\maketag@@@{(\ignorespaces #1\unskip\@@italiccorr)\equcaption{#1}}}
%    \end{macrocode}
% 修改 \cs{tagform} 会影响 \cs{eqref}。
%    \begin{macrocode}
\renewcommand{\eqref}[1]{\textup{(\ref{#1})}}
%    \end{macrocode}
%
% \subsubsection{中文字体}
% \pkg{ctex}在微软下使用雅黑字体,在macOS下使用苹方字体。这里不做更改。
%
% \subsubsection{字号}
% WORD 中的字号对应该关系如下(1bp = 72.27/72 pt):
% \begin{center}
% \begin{tabular}{llll}
% \toprule
% 初号 & 42bp & 14.82mm & 42.1575pt \\
% 小初 & 36bp & 12.70mm & 36.135 pt \\
% 一号 & 26bp & 9.17mm & 26.0975pt \\
% 小一 & 24bp & 8.47mm & 24.09pt \\
% 二号 & 22bp & 7.76mm & 22.0825pt \\
% 小二 & 18bp & 6.35mm & 18.0675pt \\
% 三号 & 16bp & 5.64mm & 16.06pt \\
% 小三 & 15bp & 5.29mm & 15.05625pt \\
% 四号 & 14bp & 4.94mm & 14.0525pt \\
% 小四 & 12bp & 4.23mm & 12.045pt \\
% 五号 & 10.5bp & 3.70mm & 10.59375pt \\
% 小五 & 9bp & 3.18mm & 9.03375pt \\
% 六号 & 7.5bp & 2.56mm & \\
% 小六 & 6.5bp & 2.29mm & \\
% 七号 & 5.5bp & 1.94mm & \\
% 八号 & 5bp & 1.76mm & \\\bottomrule
% \end{tabular}
% \end{center}
%
% \begin{macro}{\normalsize}
% 默认正文小四号 (12bp) 字,行距为固定值 20 bp。
%    \begin{macrocode}
\renewcommand\normalsize{%
  \@setfontsize\normalsize{12bp}{20bp}%
  \abovedisplayskip=12bp \@plus 2bp \@minus 2bp
  \abovedisplayshortskip=12bp \@plus 2bp \@minus 2bp
  \belowdisplayskip=\abovedisplayskip
  \belowdisplayshortskip=\abovedisplayshortskip}
%    \end{macrocode}
% \end{macro}
%
% \begin{macro}{\hw@def@fontsize}
% 根据习惯定义字号。用法:
% \cs{hw@def@fontsize}\marg{字号名称}\marg{磅数}
%
% 避免了字号选择和行距的紧耦合。所有字号定义时为单倍行距,并提供选项指定行距倍数。
%    \begin{macrocode}
\def\hw@def@fontsize#1#2{%
  \expandafter\newcommand\csname #1\endcsname[1][1.3]{%
    \fontsize{#2}{##1\dimexpr #2}\selectfont}}
%    \end{macrocode}
% \end{macro}
%
% \begin{macro}{\chuhao}
% \begin{macro}{\xiaochu}
% \begin{macro}{\yihao}
% \begin{macro}{\xiaoyi}
% \begin{macro}{\erhao}
% \begin{macro}{\xiaoer}
% \begin{macro}{\sanhao}
% \begin{macro}{\xiaosan}
% \begin{macro}{\sihao}
% \begin{macro}{\banxiaosi}
% \begin{macro}{\xiaosi}
% \begin{macro}{\dawu}
% \begin{macro}{\wuhao}
% \begin{macro}{\xiaowu}
% \begin{macro}{\liuhao}
% \begin{macro}{\xiaoliu}
% \begin{macro}{\qihao}
% \begin{macro}{\bahao}
% 一组字号定义。
%    \begin{macrocode}
\hw@def@fontsize{chuhao}{42bp}
\hw@def@fontsize{xiaochu}{36bp}
\hw@def@fontsize{yihao}{26bp}
\hw@def@fontsize{xiaoyi}{24bp}
\hw@def@fontsize{erhao}{22bp}
\hw@def@fontsize{xiaoer}{18bp}
\hw@def@fontsize{sanhao}{16bp}
\hw@def@fontsize{xiaosan}{15bp}
\hw@def@fontsize{sihao}{14bp}
\hw@def@fontsize{banxiaosi}{13bp}
\hw@def@fontsize{xiaosi}{12bp}
\hw@def@fontsize{dawu}{11bp}
\hw@def@fontsize{wuhao}{10.5bp}
\hw@def@fontsize{xiaowu}{9bp}
\hw@def@fontsize{liuhao}{7.5bp}
\hw@def@fontsize{xiaoliu}{6.5bp}
\hw@def@fontsize{qihao}{5.5bp}
\hw@def@fontsize{bahao}{5bp}
%    \end{macrocode}
% \end{macro}
% \end{macro}
% \end{macro}
% \end{macro}
% \end{macro}
% \end{macro}
% \end{macro}
% \end{macro}
% \end{macro}
% \end{macro}
% \end{macro}
% \end{macro}
% \end{macro}
% \end{macro}
% \end{macro}
% \end{macro}
% \end{macro}
% \end{macro}
%
%
% \subsection{中文标点}
%
% \newcommand\unicodechar[1]{U+#1(\symbol{"#1})}
% 由于 Unicode 的一些标点符号是中西文混用的:
% \unicodechar{00B7}、
% \unicodechar{2013}、
% \unicodechar{2014}、
% \unicodechar{2018}、
% \unicodechar{2019}、
% \unicodechar{201C}、
% \unicodechar{201D}、
% \unicodechar{2025}、
% \unicodechar{2026}、
% \unicodechar{2E3A},
% 所以要根据语言设置正确的字体。
% \footnote{\url{https://gihwb.com/CTeX-org/ctex-kit/issues/389}}
% 所以要根据语言设置正确的字体。
%    \begin{macrocode}
\newcommand\hw@setchinese{%
  \xeCJKResetPunctClass
}
\newcommand\hw@setenglish{%
  \xeCJKDeclareCharClass{HalfLeft}{"2018, "201C}%
  \xeCJKDeclareCharClass{HalfRight}{
    "00B7, "2019, "201D, "2013, "2014, "2025, "2026, "2E3A,
  }%
}
\newcommand\hw@setdefaultlanguage{%
  \ifhw@chinese
    \hw@setchinese
  \else
    \hw@setenglish
  \fi
}
%    \end{macrocode}
%
% \subsection{局部设置}
% \subsubsection{页眉页脚}
% \label{sec:headerfooter}
%
% 定义页眉和页脚样式。
% \begin{macro}{\ps@hw@empty}
% \begin{macro}{\ps@hw@plain}
% \begin{macro}{\ps@hw@headings}
% \begin{itemize}
% \item \texttt{hw@empty}:页眉页脚都没有
% \item \texttt{hw@plain}:只显示页脚的页码。\cs{chapter} 自动调用
% \cs{thispagestyle\{hw@plain\}}。
% \item \texttt{hw@headings}:页眉页脚同时显示
% \end{itemize}
%    \begin{macrocode}
\fancypagestyle{hw@empty}{%
  \fancyhf{}
  \renewcommand{\headrulewidth}{0pt}
  \renewcommand{\footrulewidth}{0pt}}
\fancypagestyle{hw@plain}{%
  \fancyhead{}
  \fancyfoot[C]{\xiaowu\thepage}
  \renewcommand{\headrulewidth}{0pt}
  \renewcommand{\footrulewidth}{0pt}}
\fancypagestyle{hw@headings}{%
  \fancyhead{}
  \fancyhead[C]{\wuhao\normalfont\leftmark}
  \fancyfoot{}
  \fancyfoot[C]{\wuhao\thepage}
  \renewcommand{\headrulewidth}{0.4pt}
  \renewcommand{\footrulewidth}{0pt}}
%    \end{macrocode}
% \end{macro}
% \end{macro}
% \end{macro}
%
% \subsubsection{段落}
% \label{sec:paragraph}
%
% 全文首行缩进 2 字符,标点符号用全角
%    \begin{macrocode}
\ctexset{%
  punct=quanjiao,
  space=auto,
  autoindent=true}
%    \end{macrocode}
%
% \subsubsection{列表}
% 利用 \pkg{enumitem} 命令调整默认列表环境间的距离,以符合中文习惯。
%    \begin{macrocode}
\RequirePackage[shortlabels]{enumitem}
\RequirePackage{environ}
\setlist{nosep}
%    \end{macrocode}
%
%
% \subsubsection{脚注}
% \label{sec:footnote}
% 脚注符合中文习惯,数字带圈。
%    \begin{macrocode}
\ifthenelse{\equal{\hw@fontset}{mac}}{
  \newfontfamily\hw@circlefont{Songti SC Light}
}{
  \ifthenelse{\equal{\hw@fontset}{windows}}{
    \newfontfamily\hw@circlefont{SimSun}
  }{
    \IfFontExistsTF{XITS-Regular.otf}{
      \newfontfamily\hw@circlefont{XITS-Regular.otf}
    }{
      \newfontfamily\hw@circlefont{xits-regular.otf}
    }
  }
}
\def\hw@textcircled#1{%
  \ifnum\value{#1} >9%
    \ClassError{hwthesis}%
      {Too many footnotes in this page.}{Keep footnote less than 10.}%
  \fi
  {\hw@circlefont\symbol{\numexpr\value{#1}+"245F\relax}}%
}
\renewcommand{\thefootnote}{\hw@textcircled{footnote}}
\renewcommand{\thempfootnote}{\hw@textcircled{mpfootnote}}
%    \end{macrocode}
%
% 定义脚注分割线,字号(宋体小五),以及悬挂缩进(1.5字符)。
%    \begin{macrocode}
\def\footnoterule{\vskip-3\p@\hrule\@width0.3\textwidth\@height0.4\p@\vskip2.6\p@}
\let\hw@footnotesize\footnotesize
\renewcommand\footnotesize{\hw@footnotesize\xiaowu[1.5]}
%\footnotemargin1.5em\relax
%    \end{macrocode}
%
% \cs{@makefnmark} 默认是上标样式,而在脚注部分要求为正文大小。利用\cs{patchcmd} 动态调整 \cs{@makefnmark} 的定义。
%    \begin{macrocode}
\let\hw@makefnmark\@makefnmark
\def\hw@@makefnmark{\hbox{{\normalfont\@thefnmark}}}
\pretocmd{\@makefntext}{\let\@makefnmark\hw@@makefnmark}{}{}
\apptocmd{\@makefntext}{\let\@makefnmark\hw@makefnmark}{}{}
%</cls>
%    \end{macrocode}
%
%
% \subsubsection{定理环境}
% \label{sec:equation}
%
% 定理标题使用黑体,正文使用宋体,冒号隔开。
%    \begin{macrocode}
%<*cfg>
\theorembodyfont{\normalfont}
\theoremheaderfont{\normalfont\heiti}
\theoremsymbol{\ensuremath{\square}}
\newtheorem*{proof}{证明}
\theoremstyle{plain}
\theoremsymbol{}
\theoremseparator{:}
\ifhw@chinese
  \newcommand\hw@assumption@name{假设}
  \newcommand\hw@definition@name{定义}
  \newcommand\hw@proposition@name{命题}
  \newcommand\hw@lemma@name{引理}
  \newcommand\hw@theorem@name{定理}
  \newcommand\hw@axiom@name{公理}
  \newcommand\hw@corollary@name{推论}
  \newcommand\hw@exercise@name{练习}
  \newcommand\hw@example@name{例}
  \newcommand\hw@remark@name{注释}
  \newcommand\hw@problem@name{问题}
  \newcommand\hw@conjecture@name{猜想}
\else
  \newcommand\hw@assumption@name{Assumption}
  \newcommand\hw@definition@name{Definition}
  \newcommand\hw@proposition@name{Proposition}
  \newcommand\hw@lemma@name{Lemma}
  \newcommand\hw@theorem@name{Theorem}
  \newcommand\hw@axiom@name{Axiom}
  \newcommand\hw@corollary@name{Corollary}
  \newcommand\hw@exercise@name{Exercise}
  \newcommand\hw@example@name{Example}
  \newcommand\hw@remark@name{Remark}
  \newcommand\hw@problem@name{Problem}
  \newcommand\hw@conjecture@name{Conjecture}
\fi
\newtheorem{assumption}{\hw@assumption@name}[chapter]
\newtheorem{definition}{\hw@definition@name}[chapter]
\newtheorem{proposition}{\hw@proposition@name}[chapter]
\newtheorem{lemma}{\hw@lemma@name}[chapter]
\newtheorem{theorem}{\hw@theorem@name}[chapter]
\newtheorem{axiom}{\hw@axiom@name}[chapter]
\newtheorem{corollary}{\hw@corollary@name}[chapter]
\newtheorem{exercise}{\hw@exercise@name}[chapter]
\newtheorem{example}{\hw@example@name}[chapter]
\newtheorem{remark}{\hw@remark@name}[chapter]
\newtheorem{problem}{\hw@problem@name}[chapter]
\newtheorem{conjecture}{\hw@conjecture@name}[chapter]
%</cfg>
%    \end{macrocode}
%
% \subsubsection{浮动对象}
% \label{sec:float}
% 设置浮动对象和文字之间的距离
%    \begin{macrocode}
%<*cls>
\setlength{\floatsep}{20bp \@plus4pt \@minus1pt}
\setlength{\intextsep}{20bp \@plus4pt \@minus2pt}
\setlength{\textfloatsep}{20bp \@plus4pt \@minus2pt}
\setlength{\@fptop}{0bp \@plus1.0fil}
\setlength{\@fpsep}{12bp \@plus2.0fil}
\setlength{\@fpbot}{0bp \@plus1.0fil}
%    \end{macrocode}
%
% 下面这组命令使浮动对象的缺省值稍微宽松一点,从而防止幅度对象占据过多的文本页面,
% 也可以防止在很大空白的浮动页上放置很小的图形。
%    \begin{macrocode}
\renewcommand{\textfraction}{0.15}
\renewcommand{\topfraction}{0.85}
\renewcommand{\bottomfraction}{0.65}
\renewcommand{\floatpagefraction}{0.60}
%    \end{macrocode}
%
% 定制浮动图形和表格标题样式
% \begin{itemize}
%   \item 图表标题字体为 11pt, 这里写作大五号
%   \item 去掉图表号后面的冒号。图序与图名文字之间空一个汉字符宽度。
%   \item 图:caption 在下,段前空 6 磅,段后空 12 磅
%   \item 表:caption 在上,段前空 12 磅,段后空 6 磅
% \end{itemize}
%
%    \begin{macrocode}
\let\old@tabular\@tabular
\def\hw@tabular{\dawu[1.5]\old@tabular}
\DeclareCaptionLabelFormat{hw}{{\dawu[1.5]\normalfont #1~#2}}
\DeclareCaptionLabelSeparator{hw}{\hspace{1em}}
\DeclareCaptionFont{hw}{\dawu[1.5]}
\captionsetup{labelformat=hw,labelsep=hw,font=hw,skip=6bp}
\captionsetup[table]{position=top}
\captionsetup[figure]{position=bottom}
\captionsetup[sub]{font=hw}
\renewcommand{\thesubfigure}{(\alph{subfigure})}
\renewcommand{\thesubtable}{(\alph{subtable})}
% \renewcommand{\p@subfigure}{:}
%    \end{macrocode}
% 我们采用 \pkg{longtable} 来处理跨页的表格。同样我们需要设置其默认字体为五号。
%    \begin{macrocode}
\let\hw@LT@array\LT@array
\def\LT@array{\dawu[1.5]\hw@LT@array} % set default font size
%    \end{macrocode}
%
% \begin{macro}{\hlinewd}
% 简单的表格使用三线表推荐用 \cs{hlinewd}。如果表格比较复杂还是用 \pkg{booktabs} 的命令好一些。
%    \begin{macrocode}
\def\hlinewd#1{%
  \noalign{\ifnum0=`}\fi\hrule \@height #1 \futurelet
    \reserved@a\@xhline}
%</cls>
%    \end{macrocode}
% \end{macro}
%
%
% \subsubsection{章节标题}
% \label{sec:theor}
%    \begin{macrocode}
%<*cfg>
\ifhw@chinese
  \ctexset{%
    chapter/name={第,章},
    appendixname=附录,
    contentsname={目\hspace{\ccwd}录},
    listfigurename=插图索引,
    listtablename=表格索引,
    figurename=图,
    tablename=表,
    bibname=参考文献,
    indexname=索引,
  }
  \newcommand\listequationname{公式索引}
  \newcommand\equationname{公式}
\else
  \newcommand\listequationname{List of Equations}
  \newcommand\equationname{Equation}
\fi
\newcommand{\cabstractname}{摘\hspace{\ccwd}要}
\newcommand{\eabstractname}{Abstract}
\let\CJK@todaysave=\today
\def\CJK@todaysmall@short{\the\year 年 \the\month 月}
\def\CJK@todaysmall{\the\year 年 \the\month 月 \the\day 日}
\def\CJK@todaybig@short{\zhdigits{\the\year}年\zhnumber{\the\month}月}
\def\CJK@todaybig{\zhdigits{\the\year}年\zhnumber{\the\month}月\zhnumber{\the\day}日}
\def\CJK@today{\CJK@todaysmall}
\renewcommand\today{\CJK@today}
\newcommand\CJKtoday[1][1]{%
  \ifcase#1\def\CJK@today{\CJK@todaysave}
    \or\def\CJK@today{\CJK@todaysmall}
    \or\def\CJK@today{\CJK@todaybig}
  \fi}
%</cfg>
%    \end{macrocode}
%
% \pkg{fancyhdr} 定义页眉页脚很方便,但是有一个非常隐蔽的坑。通过 \pkg{fancyhdr}
% 定义的样式在第一次被调用时会修改 \cs{chaptermark},这会导致页眉信息错误(多余
% 章号并且英文大写)。这是因为在原始的 \file{book.cls} 中定义如下(大意):
% \begin{latex}
% \newcommand\chaptername{Chapter}
% \newcommand\@chapapp{\chaptername}
% \def\chaptermark#1{
%   \markboth{\MakeUppercase{\@chapapp\ \thechapter}}{}}
% \end{latex}
% 很显然这个 \cs{\@chapapp} 不适合中文,因此我们使用\cs{CTEXthechapter}(
% 如,“第 x 章”),同时会将 \cs{MakeUppercase} 去掉。也就是说我们会做如下动作:
% \begin{latex}
% \renewcommand{\chaptermark}[1]{\@mkboth{\CTEXthechapter\hskip\ccwd#1}{}}
% \end{latex}
% 但,\pkg{fancyhdr} 不知何故在 \cs{ps@fancy} 中对 \cs{chaptermark} 进行重定义
% (其实一模一样),而这个 \cs{ps@fancy} 会在 \cs{fancypagestyle} 中使用,如下:
% \begin{latex}
% \newcommand{\fancypagestyle}[2]{%
%   \@namedef{ps@#1}{\let\fancy@gbl\relax#2\relax\ps@fancy}}
% \end{latex}
% 这样的话,\cs{ps@fancy} 会在 \pkg{fancyhdr} 定义的任何样式首次样被激活时调用,从
% 而覆盖我们的 \cs{chaptermark} 定义(后续样式再激活不会重复覆盖)。所以我们采用如下
% 方法解决:
%    \begin{macrocode}
%<*cls>
\AtBeginDocument{%
  \pagestyle{hw@empty}
  \renewcommand{\chaptermark}[1]{\@mkboth{\CTEXthechapter\hskip\ccwd#1}{}}}
%    \end{macrocode}
%
% 各级标题格式设置。
% \begin{description}
% \item[chapter] 章序号与章名之间空一个汉字符 黑体三号字,居中书写,单倍行距,段
%   前空 24 磅,段后空 18 磅。本科要求:段前段后间距 30/20 pt,行距 20pt。但正文
%   章节 30pt 的话和样例效果不一致。
% \item[section] 一级节标题,例如:\fbox{2.1 实验装置与实验方法}。节标题序号与标
%   题名之间空一个汉字符(下同)。采用黑体四号(14pt)字居左书写,行距为固定
%   值 20 磅,段前空 24 磅,段后空 6 磅。本科:25/12 pt,行距 18pt。
% \item[subsection] 二级节标题,例如:\fbox{2.1.1 实验装置}。采用黑体 13pt 字居左
%   书写,行距为固定值 20 磅,段前空 12 磅,段后空 6 磅。本科:中文黑体 12pt 字,
%   英文 13pt 字,段间距 12/6 pt,行距 15pt。
% \item[subsubsection] 三级节标题,例如:\fbox{2.1.2.1 归纳法}。采用黑体小四号
%   (12pt)字居左书写,行距为固定值 20 磅,段前空 12 磅,段后空 6 磅。
%
% \end{description}
%    \begin{macrocode}
\newcommand\hw@chapter@titleformat[1]{%
    \ifthenelse%
      {\equal{#1}{\eabstractname}}%
      {\bfseries #1}%
      {#1}%
  }
\ctexset{%
  chapter={
    afterindent=true,
    pagestyle={hw@headings},
    beforeskip={9bp},
    aftername=\hskip\ccwd,
    afterskip={24bp},
    format={\centering\sffamily\sanhao[1]},
    nameformat=\relax,
    numberformat=\relax,
    titleformat=\hw@chapter@titleformat,
    lofskip=0pt,
    lotskip=0pt,
  },
  section={
    afterindent=true,
    beforeskip={24bp\@plus 1ex \@minus .2ex},
    afterskip={6bp\@plus .2ex},
    format={\sffamily\sihao[1.429]},
  },
  subsection={
    afterindent=true,
    beforeskip={16bp\@plus 1ex \@minus .2ex},
    afterskip={6bp \@plus .2ex},
    format={\sffamily\banxiaosi[1.538]},
    numberformat={\sffamily\banxiaosi[1.538]},
  },
  subsubsection={
    afterindent=true,
    beforeskip={16bp\@plus 1ex \@minus .2ex},
    afterskip={6bp \@plus .2ex},
    format={\sffamily\xiaosi[1.667]},
  },
  paragraph/afterindent=true,
  subparagraph/afterindent=true}
%    \end{macrocode}
%
% \begin{macro}{\hw@chapter*}
% 默认的 \cs{chapter*} 很难同时满足研究生院和本科生的论文要求。本科论文要求所有的
% 章都出现在目录里,比如摘要、Abstract、主要符号表等,所以可以简单的扩展默
% 认\cs{chapter*} 实现这个目的。但是研究生又不要这些出现在目录中,而且致谢和声明
% 部分的章名、页眉和目录都不同,所以定义一个灵活的 \cs{hw@chapter*} 专门处理这些
% 要求。
%
% \cs{hw@chapter*}\oarg{tocline}\marg{title}\oarg{header}: tocline 是出现在目录
% 中的条目,如果为空则此 chapter 不出现在目录中,如果省略表示目录出现 title;
% title 是章标题;header 是页眉出现的标题,如果忽略则取 title。通过这个宏我才真
% 正体会到 \TeX\ macro 的力量!
%    \begin{macrocode}
\newcounter{hw@bookmark}
\NewDocumentCommand\hw@chapter{s o m o}{
  \IfBooleanF{#1}{%
    \ClassError{hwthesis}{You have to use the star form: \string\hw@chapter*}{}
  }%
  \if@openright\cleardoublepage\else\clearpage\fi\phantomsection%
  \IfValueTF{#2}{%
    \ifthenelse{\equal{#2}{}}{%
      \addtocounter{hw@bookmark}\@ne
      \pdfbookmark[0]{#3}{hwchapter.\thehw@bookmark}
    }{%
      \addcontentsline{toc}{chapter}{#3}
    }
  }{%
    \addcontentsline{toc}{chapter}{#3}
  }%
  \ctexset{chapter/beforeskip=25bp}
  \chapter*{#3}%
  \ctexset{chapter/beforeskip=15bp}
  \IfValueTF{#4}{%
    \ifthenelse{\equal{#4}{}}
    {\@mkboth{}{}}
    {\@mkboth{#4}{#4}}
  }{%
    \@mkboth{#3}{#3}
  }
}
%    \end{macrocode}
% \end{macro}
%
%
% \subsubsection{目录}
% \label{sec:toc}
% 最多 4 层,即: x.x.x.x,对应的命令和层序号分别是:
% \cs{chapter}(0), \cs{section}(1), \cs{subsection}(2), \cs{subsubsection}(3)。
% \changes{v3.1}{2007/10/09}{博士论文目录只出现到第 3 级标题即可。}
% \changes{v5.0.0}{2015/05/21}{硕士博士论文目录只出现到第 3 级标题即可。其他未明确要求。}
%    \begin{macrocode}
\setcounter{secnumdepth}{3}
\setcounter{tocdepth}{2}
%    \end{macrocode}
%
% 每章标题行前空 6 磅,后空 0 磅。章节名中英文用 Arial 字体,页码仍用 Times。
% \begin{macro}{\tableofcontents}
%    \begin{macrocode}
\renewcommand\tableofcontents{%
  \hw@chapter*[]{\contentsname}
  \xiaosi[1.65]\@starttoc{toc}\normalsize}
%    \end{macrocode}
% 调整目录样式,允许指定目录字体。
% \changes{v5.2.2}{2016/01/23}{用 \cs{patchcmd} 修改 \cs{@dottedtocline}。}
%    \begin{macrocode}
\def\@pnumwidth{2em}
\def\@tocrmarg{\@pnumwidth}
\def\@dotsep{1}
\renewcommand*\l@chapter[2]{%
  \ifnum \c@tocdepth >\m@ne
    \addpenalty{-\@highpenalty}%
    4bp \@plus\p@
    \setlength\@tempdima{4em}%
    \begingroup
      \parindent \z@ \rightskip \@pnumwidth
      \parfillskip -\@pnumwidth
      \leavevmode
      \advance\leftskip\@tempdima
      \hskip -\leftskip
      {#1}%
      \leaders\hbox{$\m@th\mkern \@dotsep mu\hbox{.}\mkern \@dotsep mu$}\hfill%
      \nobreak{#2}\par
      \penalty\@highpenalty
    \endgroup
  \fi}

\patchcmd{\@dottedtocline}{\hb@xt@\@pnumwidth}{\hbox}{}{}
\renewcommand*\l@section{%
  \@dottedtocline{1}{\ccwd}{2.1em}}
\renewcommand*\l@subsection{%
  \@dottedtocline{2}{2\ccwd}{3em}}
\renewcommand*\l@subsubsection{%
  \@dottedtocline{3}{3.5em}{3.8em}}
%    \end{macrocode}
% \end{macro}
%
% \subsection{附加页面}
% \label{sec:etc}
% \subsubsection{封面和封底}
% \subsubsection{摘要}
% \subsubsection{主要符号表}
% \subsubsection{致谢与声明}
%
% \subsubsection{图表索引}
% \label{sec:threeindex}
% \begin{macro}{\listoffigures}
% \begin{macro}{\listoffigures*}
% \begin{macro}{\listoftables}
% \begin{macro}{\listoftables*}
% 定义图表以及公式目录样式。
% \changes{v2.5}{2006/05/18}{增加插图、表格和公式索引。}
% \changes{v2.5}{2006/05/19}{为了让索引中能出现\textbf{图 xxx},不得不修改 \LaTeX
%   内部命令 \cs{@caption}。}
% \changes{v2.6.4}{2006/10/23}{增加 \cs{listoffigures*},\cs{listoftables*}。}
% \changes{v4.5.1}{2009/01/06}{更优雅的插图/表格索引,避免跟 \pkg{caption} 包冲
% 突。\cs{thu@listof} 相应修改。}
%    \begin{macrocode}
%<*cls>
\def\thu@starttoc#1{% #1: float type, prepend type name in \listof*** entry.
  \let\oldnumberline\numberline
  \def\numberline##1{\oldnumberline{\csname #1name\endcsname\hskip.4em ##1}}
  \@starttoc{\csname ext@#1\endcsname}
  \let\numberline\oldnumberline}
\def\thu@listof#1{% #1: float type
  \@ifstar
    {\thu@chapter*[]{\csname list#1name\endcsname}\thu@starttoc{#1}}
    {\thu@chapter*{\csname list#1name\endcsname}\thu@starttoc{#1}}}
\renewcommand\listoffigures{\thu@listof{figure}}
\renewcommand*\l@figure{\ifthu@bachelor\relax\else\addvspace{6bp}\fi\@dottedtocline{1}{0em}{4em}}
\renewcommand\listoftables{\thu@listof{table}}
\let\l@table\l@figure
%    \end{macrocode}
% \end{macro}
% \end{macro}
% \end{macro}
% \end{macro}
%
% \begin{macro}{\equcaption}
% \changes{v2.6.2}{2006/06/19}{此命令配合 \pkg{amsmath} 命令基本可以满足所有
% 公式需要。}
%   本命令只是为了生成公式列表,所以这个 caption 是假的。如果要编号最好用
%    equation 环境,如果是其它编号环境,请手动添加 \cs{equcaption}。
% 用法如下:
%
% \cs{equcaption}\marg{counter}
%
% \marg{counter} 指定出现在索引中的编号,一般取 \cs{theequation},如果你是用
%  \pkg{amsmath} 的 \cs{tag},那么默认是 \cs{tag} 的参数;除此之外可能需要你
% 手工指定。
%
% \changes{v2.5}{2006/05/19}{将公式编号写入临时文件以便生成公式列表。}
% \changes{v2.5.3}{2006/06/03}{取消 \cs{equcaption} 的参数}
%    \begin{macrocode}
\def\ext@equation{loe}
\def\equcaption#1{%
  \addcontentsline{\ext@equation}{equation}%
                  {\protect\numberline{#1}}}
%    \end{macrocode}
% \end{macro}
%
% \begin{macro}{\listofequations}
% \begin{macro}{\listofequations*}
% \LaTeX\ 默认没有公式索引,此处定义自己的 \cs{listofequations}。
% \changes{v2.5}{2006/05/19}{增加公式索引命令。}
% \changes{v2.5.1}{2006/05/26}{公式索引项 numwidth 增加。}
% \changes{v2.6.4}{2006/10/23}{增加 \cs{listofequations*}。}
%    \begin{macrocode}
\newcommand\listofequations{\thu@listof{equation}}
\let\l@equation\l@figure
%</cls>
%    \end{macrocode}
% \end{macro}
% \end{macro}
%
% \subsection{参考文献}
% \label{sec:ref}
%
% \changes{v5.4.4}{2018/04/12}{参考文献列表的页码使用 hyphen 取代 en dash。}
%
% \begin{macro}{\inlinecite}
% 依赖于 \pkg{natbib} 宏包,修改其中的命令。 旧命令 \cs{onlinecite} 依然可用。
% \changes{v5.0.0}{2015/11/23}{用 \cs{inlinecite} 替换 \cs{onlinecite}。为保证兼
% 容性,\cs{onlinecite} 会保留。}
%    \begin{macrocode}
%<*cls>
\newcommand\bibstyle@inline{\bibpunct{[}{]}{,}{n}{,}{,}}
\DeclareRobustCommand\inlinecite{\@inlinecite}
\def\@inlinecite#1{\begingroup\let\@cite\NAT@citenum\citep{#1}\endgroup}
\let\onlinecite\inlinecite
%</cls>
%    \end{macrocode}
% \end{macro}
%
% 参考文献的正文部分用五号字。
% 行距采用固定值 16 磅,段前空 3 磅,段后空 0 磅。
% 本科生要求固定行距 17pt,段前后间距 3pt。
%
% \begin{macro}{\thumasterbib}
% \begin{macro}{\thuphdbib}
%   本科生和研究生模板要求外文硕士论文参考文献显示``[Master Thesis]'',而博士模板
%   则于 2007 年冬要求显示为``[M]''。对应的外文博士论文参考文献分别显示为``[Phd
%   Thesis]''和``[D]''。
%   研究生写作指南(201109)要求:
%   中文硕士学位论文标注``[硕士学位论文]'',
%   中文博士学位论文标注``[博士学位论文]'',外文学位论文标注``[D]''。
%   本科生写作指南未指定,参考文献著录格式文档中对中外文学位论文都标注``[D]''。
% \changes{v4.7}{2012/05/29}{修改两个宏使其对应不同的中文论文需求。}
%    \begin{macrocode}
%<*cfg>
\def\thumasterbib{\ifthu@bachelor D\else 硕士学位论文\fi}
\def\thuphdbib{\ifthu@bachelor D\else 博士学位论文\fi}
%</cfg>
%    \end{macrocode}
% \end{macro}
% \end{macro}
%
% 复用 \pkg{natbib} 的 \texttt{thebibliography} 环境,调整距离。
% \changes{v2.4}{2006/04/15}{参考文献间距调小一点,label 长度增加一点,以便让超过
%  100 的参考文献更好地对齐。}
% \changes{v2.5}{2006/05/13}{参考文献序号靠左,而不是靠右。}
% \changes{v2.6.4}{2006/10/23}{调整参考文献标签宽度,使得条目增多时仍能对齐。}
% \changes{v5.4.0}{2017/12/03}{基于 \pkg{natbib} 的环境调整距离兼容性更好。}
% \changes{v5.4.4}{2018/04/14}{参考文献标号左对齐。}
%    \begin{macrocode}
%<*cls>
\renewcommand\bibsection{\thu@chapter*{\bibname}}
\renewcommand\bibfont{\ifthu@bachelor\wuhao[1.619]\else\wuhao[1.5]\fi}
\setlength\bibhang{2\ccwd}
\addtolength{\bibsep}{-0.7em}
\setlength{\labelsep}{0.4em}
\def\@biblabel#1{[#1]\hfill}
%    \end{macrocode}
%
% 两种引用样式:
% \changes{v5.4.0}{2017/12/3}{\cs{bibliographystyle}\marg{newbib} will cause \cs{bibstyle@newbib} to
% be called on THE NEXT LATEX RUN (via the aux file).}
% \changes{v5.4.1}{2017/12/04}{bst 在 ctan 上不分路径,故加前缀。}
%    \begin{macrocode}
\expandafter\newcommand\csname bibstyle@thuthesis-numeric\endcsname{%
  \bibpunct{[}{]}{,}{s}{,}{\textsuperscript{,}}}
\expandafter\newcommand\csname bibstyle@thuthesis-author-year\endcsname{%
  \bibpunct{(}{)}{;}{a}{,}{,}}
%    \end{macrocode}
%
% 下面修改 \pkg{natbib} 的引用格式,主要是将页码写在上标位置。
% numeric 模式的 \cs{citet} 的页码:
%    \begin{macrocode}
\patchcmd\NAT@citexnum{%
  \@ifnum{\NAT@ctype=\z@}{%
    \if*#2*\else\NAT@cmt#2\fi
  }{}%
  \NAT@mbox{\NAT@@close}%
}{%
  \NAT@mbox{\NAT@@close}%
  \@ifnum{\NAT@ctype=\z@}{%
    \if*#2*\else\textsuperscript{#2}\fi
  }{}%
}{}{}
%    \end{macrocode}
%
% Numeric 模式的 \cs{citep} 的页码:
%    \begin{macrocode}
\renewcommand\NAT@citesuper[3]{\ifNAT@swa
  \if*#2*\else#2\NAT@spacechar\fi
\unskip\kern\p@\textsuperscript{\NAT@@open#1\NAT@@close\if*#3*\else#3\fi}%
   \else #1\fi\endgroup}
%    \end{macrocode}
%
% Author-year 模式的 \cs{citet} 的页码:
%    \begin{macrocode}
\patchcmd{\NAT@citex}{%
  \if*#2*\else\NAT@cmt#2\fi
  \if\relax\NAT@date\relax\else\NAT@@close\fi
}{%
  \if\relax\NAT@date\relax\else\NAT@@close\fi
  \if*#2*\else\textsuperscript{#2}\fi
}{}{}
%    \end{macrocode}
%
% Author-year 模式的 \cs{citep} 的页码:
%    \begin{macrocode}
\renewcommand\NAT@cite%
    [3]{\ifNAT@swa\NAT@@open\if*#2*\else#2\NAT@spacechar\fi
        #1\NAT@@close\if*#3*\else\textsuperscript{#3}\fi\else#1\fi\endgroup}
%    \end{macrocode}
%
% 在顺序编码制下,\pkg{natbib} 只有在三个以上连续文献引用才会使用连接号,
% 这里修改为允许两个引用使用连接号。
% \changes{v5.4.4}{2018/04/12}{允许连续两个文献引用使用连接号。}
%    \begin{macrocode}
\patchcmd{\NAT@citexnum}{%
  \ifx\NAT@last@yr\relax
    \def@NAT@last@yr{\@citea}%
  \else
    \def@NAT@last@yr{--\NAT@penalty}%
  \fi
}{%
  \def@NAT@last@yr{-\NAT@penalty}%
}{}{}
%</cls>
%    \end{macrocode}
%
% \subsubsection{附录}
% \subsubsection{个人简历}
%
%
% \subsection{结束部分}
% \label{sec:finish}
%    \begin{macrocode}
\AtEndOfClass{%%
%% This is file `njurepo.cfg',
%% generated with the docstrip utility.
%%
%% The original source files were:
%%
%% njurepo.dtx  (with options: `cfg')
%% 
%% This is a generated file.
%% 
%% Copyright (C) 2019 by Zangwei Zheng <zhengzangw@gmail.com>
%% 
%% This file may be distributed and/or modified under the
%% conditions of the LaTeX Project Public License, either version 1.3
%% of this license or (at your option) any later version.
%% The latest version of this license is in:
%%   http://www.latex-project.org/lppl.txt
%% and version 1.3 or later is part of all distributions of LaTeX
%% version 2005/12/01 or later.
%% 

\ProvidesFile{njurepo.cfg}
[2019/01/25 1.0.0 Nanjing University Report Template]
\theorembodyfont{\normalfont}
\theoremheaderfont{\normalfont\heiti}
\theoremsymbol{\ensuremath{\square}}
\newtheorem*{proof}{证明}
\theoremstyle{plain}
\theoremsymbol{}
\theoremseparator{:}
\ifhw@chinese
  \newcommand\hw@assumption@name{假设}
  \newcommand\hw@definition@name{定义}
  \newcommand\hw@proposition@name{命题}
  \newcommand\hw@lemma@name{引理}
  \newcommand\hw@theorem@name{定理}
  \newcommand\hw@axiom@name{公理}
  \newcommand\hw@corollary@name{推论}
  \newcommand\hw@exercise@name{练习}
  \newcommand\hw@example@name{例}
  \newcommand\hw@remark@name{注释}
  \newcommand\hw@problem@name{问题}
  \newcommand\hw@conjecture@name{猜想}
\else
  \newcommand\hw@assumption@name{Assumption}
  \newcommand\hw@definition@name{Definition}
  \newcommand\hw@proposition@name{Proposition}
  \newcommand\hw@lemma@name{Lemma}
  \newcommand\hw@theorem@name{Theorem}
  \newcommand\hw@axiom@name{Axiom}
  \newcommand\hw@corollary@name{Corollary}
  \newcommand\hw@exercise@name{Exercise}
  \newcommand\hw@example@name{Example}
  \newcommand\hw@remark@name{Remark}
  \newcommand\hw@problem@name{Problem}
  \newcommand\hw@conjecture@name{Conjecture}
\fi
\newtheorem{assumption}{\hw@assumption@name}[chapter]
\newtheorem{definition}{\hw@definition@name}[chapter]
\newtheorem{proposition}{\hw@proposition@name}[chapter]
\newtheorem{lemma}{\hw@lemma@name}[chapter]
\newtheorem{theorem}{\hw@theorem@name}[chapter]
\newtheorem{axiom}{\hw@axiom@name}[chapter]
\newtheorem{corollary}{\hw@corollary@name}[chapter]
\newtheorem{exercise}{\hw@exercise@name}[chapter]
\newtheorem{example}{\hw@example@name}[chapter]
\newtheorem{remark}{\hw@remark@name}[chapter]
\newtheorem{problem}{\hw@problem@name}[chapter]
\newtheorem{conjecture}{\hw@conjecture@name}[chapter]
\ifhw@chinese
  \ctexset{%
    chapter/name={第,章},
    appendixname=附录,
    contentsname={目\hspace{\ccwd}录},
    listfigurename=插图索引,
    listtablename=表格索引,
    figurename=图,
    tablename=表,
    bibname=参考文献,
    indexname=索引,
  }
  \newcommand\listequationname{公式索引}
  \newcommand\equationname{公式}
\else
  \newcommand\listequationname{List of Equations}
  \newcommand\equationname{Equation}
\fi
\newcommand{\cabstractname}{摘\hspace{\ccwd}要}
\newcommand{\eabstractname}{Abstract}
\let\CJK@todaysave=\today
\def\CJK@todaysmall@short{\the\year 年 \the\month 月}
\def\CJK@todaysmall{\the\year 年 \the\month 月 \the\day 日}
\def\CJK@todaybig@short{\zhdigits{\the\year}年\zhnumber{\the\month}月}
\def\CJK@todaybig{\zhdigits{\the\year}年\zhnumber{\the\month}月\zhnumber{\the\day}日}
\def\CJK@today{\CJK@todaysmall}
\renewcommand\today{\CJK@today}
\newcommand\CJKtoday[1][1]{%
  \ifcase#1\def\CJK@today{\CJK@todaysave}
    \or\def\CJK@today{\CJK@todaysmall}
    \or\def\CJK@today{\CJK@todaybig}
  \fi}
\endinput
%%
%% End of file `njurepo.cfg'.
}
\AtEndOfClass{\sloppy}
%    \end{macrocode}
%</cls> 
%
%
%
%
%
%
%
% \iffalse
%    \begin{macrocode}
%<*dtx-style>
\ProvidesPackage{dtx-style}
\RequirePackage{hypdoc}
\RequirePackage{ifthen}
\RequirePackage[UTF8,scheme=chinese]{ctex}
\RequirePackage{newpxtext}
\RequirePackage{newpxmath}
\RequirePackage[
  top=2.5cm, bottom=2.5cm,
  left=4cm, right=2cm,
  headsep=3mm]{geometry}
\RequirePackage{array,longtable,booktabs}
\RequirePackage{listings}
\RequirePackage{fancyhdr}
\RequirePackage{xcolor}
\RequirePackage{enumitem}
\RequirePackage{etoolbox}
\RequirePackage{metalogo}

\ifthenelse{\equal{\@nameuse{g__ctex_fontset_tl}}{mac}}{%
  \xeCJKsetwidth{‘’“”}{1em}
}{}

\colorlet{hw@macro}{blue!60!black}
\colorlet{hw@env}{blue!70!black}
\colorlet{hw@option}{purple}
\patchcmd{\PrintMacroName}{\MacroFont}{\MacroFont\bfseries\color{hw@macro}}{}{}
\patchcmd{\PrintDescribeMacro}{\MacroFont}{\MacroFont\bfseries\color{hw@macro}}{}{}
\patchcmd{\PrintDescribeEnv}{\MacroFont}{\MacroFont\bfseries\color{hw@env}}{}{}
\patchcmd{\PrintEnvName}{\MacroFont}{\MacroFont\bfseries\color{hw@env}}{}{}

\def\DescribeOption{%
  \leavevmode\@bsphack\begingroup\MakePrivateLetters%
  \Describe@Option}
\def\Describe@Option#1{\endgroup
  \marginpar{\raggedleft\PrintDescribeOption{#1}}%
  \hw@special@index{option}{#1}\@esphack\ignorespaces}
\def\PrintDescribeOption#1{\strut \MacroFont\bfseries\sffamily\color{hw@option} #1\ }
\def\hw@special@index#1#2{\@bsphack
  \begingroup
    \HD@target
    \let\HDorg@encapchar\encapchar
    \edef\encapchar usage{%
      \HDorg@encapchar hdclindex{\the\c@HD@hypercount}{usage}%
    }%
    \index{#2\actualchar{\string\ttfamily\space#2}
           (#1)\encapchar usage}%
    \index{#1:\levelchar#2\actualchar
           {\string\ttfamily\space#2}\encapchar usage}%
  \endgroup
  \@esphack}

\lstdefinestyle{lstStyleBase}{%
   basicstyle=\small\ttfamily,
   aboveskip=\medskipamount,
   belowskip=\medskipamount,
   lineskip=0pt,
   boxpos=c,
   showlines=false,
   extendedchars=true,
   upquote=true,
   tabsize=2,
   showtabs=false,
   showspaces=false,
   showstringspaces=false,
   numbers=none,
   linewidth=\linewidth,
   xleftmargin=4pt,
   xrightmargin=0pt,
   resetmargins=false,
   breaklines=true,
   breakatwhitespace=false,
   breakindent=0pt,
   breakautoindent=true,
   columns=flexible,
   keepspaces=true,
   gobble=2,
   framesep=3pt,
   rulesep=1pt,
   framerule=1pt,
   backgroundcolor=\color{gray!5},
   stringstyle=\color{green!40!black!100},
   keywordstyle=\bfseries\color{blue!50!black},
   commentstyle=\slshape\color{black!60}}

\lstdefinestyle{lstStyleShell}{%
   style=lstStyleBase,
   frame=l,
   rulecolor=\color{purple},
   language=bash}

\lstdefinestyle{lstStyleLaTeX}{%
   style=lstStyleBase,
   frame=l,
   rulecolor=\color{violet},
   language=[LaTeX]TeX}

\lstnewenvironment{latex}{\lstset{style=lstStyleLaTeX}}{}
\lstnewenvironment{shell}{\lstset{style=lstStyleShell}}{}

\setlist{nosep}

\DeclareDocumentCommand{\option}{m}{\textsf{#1}}
\DeclareDocumentCommand{\env}{m}{\texttt{#1}}
\DeclareDocumentCommand{\pkg}{s m}{%
  \texttt{#2}\IfBooleanF#1{\hw@special@index{package}{#2}}}
\DeclareDocumentCommand{\file}{s m}{%
  \texttt{#2}\IfBooleanF#1{\hw@special@index{file}{#2}}}
\newcommand{\myentry}[1]{%
  \marginpar{\raggedleft\color{purple}\bfseries\strut #1}}
\newcommand{\note}[2][Note]{{%
  \color{magenta}{\bfseries #1}\emph{#2}}}

\def\njurepo{\textsc{NJU}\-\textsc{repo}}
\def\thuthesis{\textsc{Thu}\-\textsc{Thesis}}
%</dtx-style>
%    \end{macrocode}
% \fi
% \Finale
